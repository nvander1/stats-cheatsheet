\documentclass[10pt,landscape]{article}
\usepackage{multicol}
\usepackage{calc}
\usepackage{ifthen}
\usepackage[landscape]{geometry}
\usepackage{amsmath,amsthm,amsfonts,amssymb}
\usepackage{color,graphicx,overpic}
\usepackage{hyperref}
\usepackage{polyglossia}
% \setdefaultlanguage{french}
\usepackage{microtype}
\usepackage{fontspec} % Xelatex
\usepackage{dsfont}
\usepackage{unicode-math}
%\setmathfont{Latin Modern Math}

\newcommand*\diff{\mathop{}\!\mathrm{d}}
\newcommand*\conv{\mathop{}\!\mathrm{conv}}
\newcommand*\ri{\mathop{}\!\mathrm{ri}}
\newcommand*\aff{\mathop{}\!\mathrm{aff}}
\newcommand*\dom{\mathop{}\!\mathrm{dom}}
\newcommand*\epi{\mathop{}\!\mathrm{epi}}

% This sets page margins to .5 inch if using letter paper, and to 1cm
% if using A4 paper. (This probably isn't strictly necessary.)
% If using another size paper, use default 1cm margins.
\ifthenelse{\lengthtest { \paperwidth = 11in}}
{ \geometry{top=.5in,left=.5in,right=.5in,bottom=.5in} }
{\ifthenelse{ \lengthtest{ \paperwidth = 297mm}}
  {\geometry{top=1cm,left=1cm,right=1cm,bottom=1cm} }
  {\geometry{top=1cm,left=1cm,right=1cm,bottom=1cm} }
}

% Turn off header and footer
\pagestyle{empty}

% Redefine section commands to use less space
\makeatletter
\renewcommand{\section}{\@startsection{section}{1}{0mm}%
  {-1ex plus -.5ex minus -.2ex}%
  {0.5ex plus .2ex}%x
{\normalfont\large\bfseries}}
\renewcommand{\subsection}{\@startsection{subsection}{2}{0mm}%
  {-1explus -.5ex minus -.2ex}%
  {0.5ex plus .2ex}%
{\normalfont\normalsize\bfseries}}
\renewcommand{\subsubsection}{\@startsection{subsubsection}{3}{0mm}%
  {-1ex plus -.5ex minus -.2ex}%
  {1ex plus .2ex}%
{\normalfont\small\bfseries}}
\makeatother

% Define BibTeX command
\def\BibTeX{{\rm B\kern-.05em{\sc i\kern-.025em b}\kern-.08em
T\kern-.1667em\lower.7ex\hbox{E}\kern-.125emX}}

% Don't print section numbers
\setcounter{secnumdepth}{0}


\setlength{\parindent}{0pt}
\setlength{\parskip}{0pt plus 0.5ex}

%My Environments
\newtheorem{ex}[section]{Exemple}
\newtheorem*{defi}{Définition}
\newtheorem*{prop}{Proposition}
\newtheorem*{theorem}{Théorème}
\newtheorem*{lemme}{Lemme}
\newtheorem*{corol}{Corollaire}
% -----------------------------------------------------------------------

\begin{document}
\raggedright
\footnotesize
\begin{multicols}{3}


  % multicol parameters
  % These lengths are set only within the two main columns
  %\setlength{\columnseprule}{0.25pt}
  \setlength{\premulticols}{1pt}
  \setlength{\postmulticols}{1pt}
  \setlength{\multicolsep}{1pt}
  \setlength{\columnsep}{2pt}

  \begin{center}
    \Large{\underline{Analyse Convexe}} \\
  \end{center}

  % Start Writing here
  \section{Ensembles convexes}

  \begin{defi}
    $A$ sous-espace affine si $(1-t) x+t y \in A, \forall x,y \in A, \forall t \in \mathbb{R}$ \\
    $A$ stable par combinaison affine : $\forall x_1,\ldots,x_m \in A, \forall \lambda_1,\ldots,\lambda_m $ tels que $ \sum_{i=1}^m \lambda_i = 1 $ on a $\sum_{i=1}^m \lambda_i x_i \in A$ 
  \end{defi}

  \begin{defi}
    $A$ convexe: pareil avec $t \in [0,1]$ et $ \forall i, \lambda_i > 0 $ 
  \end{defi}

  \begin{lemme}
    Soient $a,b>0$ et $K$ convexe. Alors $a K + b K = (a+b) K$
  \end{lemme}

  \begin{defi}
    Soit $F$ sous-espace affine, $x \in F$. On à $\dim (F) = \dim ( F-x) $
  \end{defi}

  \begin{defi}
    On appelle simplexe de dimension $n$ l'enveloppe convexe de $n+1$ points.
  \end{defi}

  \begin{lemme}
    Soit $K$ un convexe de dimension $n$ alors in contient un simplexe de dimension $n+1$
  \end{lemme}

  \section{Théorème de Carathéodory et de Helly}

  \begin{theorem}[Théorème de Carathéodory]
    Soit $ A \subset \mathbb{R}^n $. Toute combinaison convexe d'éléments de $A$ peut s'écrire comme combinaison d'au plus $n+1$ éléments de $A$.
  \end{theorem}

  \begin{theorem}[Théorème de Radon] 
    Soit $S \subset \mathbb{R}^n, \left\vert{N}\right\vert \geq n+2 $, il existe $A$ et $B$ verifiant: \\
    $ A \cap B = \emptyset $ et $ \conv (A) \cap \conv (B) \neq \emptyset $
  \end{theorem}

  \begin{theorem}[Théorème de Helly]
    Si une famille de sous-ensembles convexes de $\mathbb{R}^n$ est telle que toute sous-famille de cardinal $n+1$ est d'intersection non vide alors la famille entière est d'intersection non vide.
  \end{theorem}

  \section{Propriétés topologiques des convexes}

  \begin{prop}
    Si $K$ convexe alors $ \mathring{K} $ et $ \bar{K} $ sont convexes.
  \end{prop}

  \begin{defi}
    $\ri (K) = \{ x \in U, U \text{ouvert et } U \cap \aff (K) \subset K \}$
  \end{defi}

  \begin{lemme}
    L'interieur relatif d'un convexe est convexe.
  \end{lemme}

  \begin{prop}
    Soit $K \neq \emptyset $ convexe de $ \mathbb{R}^n $, alors $ \ri (K) \neq \emptyset $ 
  \end{prop}

  \begin{defi}
    Soit $K$ convexe. On appelle jauge de $K$: \\
    $ j(x) = \inf \{ s>0 ; x \in s K \} $
  \end{defi}

  \begin{prop}
    Soit $K$ convexe contenant $0$ et $j$ sa jauge.
    \begin{enumerate}
      \item $j(tx) = t j(x) ; \forall t \geq 0$
      \item $j(x+y) \leq j(x) + j(y)$
    \end{enumerate}
  \end{prop}

  \begin{prop}
    Soit $K$ convexe tel que $ 0 \in \mathring{K} $ et $j$ sa jauge. Alors $j(x) < \infty $ et $j$ continue sur $E$.
  \end{prop}

  \begin{prop}
    Soit $K$ convexe contenant $0$. \\
    $ \mathring{K} \subset \{j<1\} \subset K \subset \{ j \leq 1 \} \subset \bar{K} $
  \end{prop}

  \begin{prop}
    Si $K$ convexe, $\mathring{K} \neq \emptyset $, $-K = K$ et $K$ ne contient pas de demi-droite alors $j$ est une norme. Et $\bar{K}$ la boule unité de $j$.
  \end{prop}

  \section{Fonctions convexes}

  \begin{prop}
    \begin{enumerate}
      \item $f$ convexe $ \implies $ $ \dom (f) = \{ x \in K ; f(x) < \infty \} $ convexe
      \item $f$ convexe $ \Leftrightarrow $ $ \epi (f) = \{ (x,t) \in \dom (f) \times \mathbb{R}; f(x) \geq t \} $ convexe
    \end{enumerate}
  \end{prop}

  \begin{lemme}
    Si $f$ convexe majorée au voisinage de $x \in E$ alors $f$ continue en $x$.
  \end{lemme}

  \begin{lemme}
    Soit $f : \mathbb{R}^n \rightarrow \mathbb{R} \cup \{ +\infty \} $ convexe. Alors $f$ localement bornée sur son domaine.
  \end{lemme}

  \begin{corol}
    En dimension finie, une fonction convexe est continue dans l'interieur de son domaine.
  \end{corol}

  \section{Théorème de Hahn-Banach}

  \begin{theorem}[Théorème de Hahn-Banach]
    Soit $E$ un espace vectoriel, $j : E \rightarrow \mathbb{R}$ une fonction 1-homogéne et sous-additive. Soit $F$ un s.e.v de $E$ et $\phi : F \rightarrow \mathbb{R} $ telle que: $ \phi (x) \leq j(x), \forall x \in F $ alors $ \exists \psi : F \rightarrow \mathbb{R} $ telle que:
    \begin{align*}
      \psi (x) &= \phi (x) , \forall x \in F \\
      \psi (x) &\leq \phi (x) , \forall x \in E
    \end{align*}
  \end{theorem}

  \begin{corol}
    Soit $E$ un espace vectoriel normé, $F$ un sous-espace et $\phi$ une forme linéaire continue sur $F$. Alors $\phi$ s'étend en une forme linéaire continue sur $E$ de même norme que $\phi$. 
  \end{corol}

  \begin{corol}
    Soit $E$ e.v.n, il existe $\phi \in E^\star$ de norme $1$ telle que $ \phi (x) = \| x \| $
  \end{corol}

  \begin{theorem}
    Soit $E$ un espace vectoriel topologique, $A$,$B$ deux ensembles convexes disjoints.
    \begin{enumerate}
      \item Si $A$ ouvert, $\exists \phi \in E^\star$ telle que $ \phi (x) < \phi (y), \forall x \in A, \forall y \in B $
      \item Si $E$ localement convexe, $A$ compact, $B$ fermé, alors $\exists \phi \in E^\star $ et $ \epsilon > 0 $ tels que $ \phi (x) + \epsilon < \phi (y) , \forall x \in A, \forall y \in B $
    \end{enumerate}

    \begin{corol}[Hyperplan d'appui]
      Soit $C$ un convexe fermé d'interieur non vide et $x \in \delta C $. Alors $C$ admet un hyperplan d'appui en $x$: $ \exists \phi \in E^\star $ non nulle telle que $ \phi (y) \leq \phi (x), \forall y \in C $
    \end{corol}


  \end{theorem}


  \section{Polarité}

  \begin{defi}
    Soit $A \subset \mathbb{R}^n$. On appelle polaire de $A$, $ A^\circ = \{ x \in \mathbb{R}^n, \langle x,y \rangle \leq 1, \forall y \in A \} $
  \end{defi}

  \begin{lemme}
    L'ensemble $A^\circ$ est convexe, férmé et contient $0$.
  \end{lemme}

  \begin{theorem}
    $A = A^{\circ \circ} \Leftrightarrow  A $ est convexe, fermé et contient $0$.
  \end{theorem}

  \begin{defi}
    Soit $A \subset \mathbb{R}^n $. On appelle fonction d'appui de $A$: $ h_K (y) = \sup\limits_{x \in A} \langle x,y \rangle $
  \end{defi}


  \begin{lemme}
    $0 \in A \implies h_A = j_{A^\circ} $ 
  \end{lemme}

  \begin{prop}
    Soit $K$ un convexe compact d'interieur non vide ( $K$ corps convexe ) tel que $K = -K$. Soit $j_K$ sa jauge. Alors $E = (\mathbb{R}^n, j_K) $ espace vectoriel normé dont la boule unité est $K$ et $ E^\star = (\mathbb{R}^n, j_{K^\circ}) $ i.e la boule unité de $E^\star$ est le polaire de $K$.
  \end{prop}

  \section{Transformée de Legendre-Fenchel}

  \begin{defi}
    $f$ est dite semi-continue inferieurement si pour toute suite convergente $(x_n)$ de $\mathbb{R}^n$ on: $ \liminf f(x_n) \geq f(\lim x_n) $
  \end{defi}

  \begin{lemme}
    $f$ s.c.i $\Leftrightarrow$ $ \epi (f) $ sous-ensemble fermé de $ \mathbb{R}^n \times \mathbb{R} $
  \end{lemme}

  \begin{lemme}
    Un $\sup$ de fonctions s.c.i est s.c.i
  \end{lemme}

  \begin{defi}[Transformée de Legendre-Fenchel]
    Soit $f$ une fonction propre. \\
    $ f^\star : y \in \mathbb{R}^n \rightarrow \sup\limits_{x \in \mathbb{R}^n} \{ \langle x, y \rangle -f(x) \} $
  \end{defi}

  \begin{lemme}
    $f^\star$ est convexe et s.c.i
  \end{lemme}

  \begin{defi}
    Soit $f$ une fonction propre et $x \in \dom (f)$. On dit que $y$ est sous-gradient de $f$ en $x$, $y \in \delta f(x) $ si: \\
    $ \langle x',y \rangle - f(x') \leq \langle x, y \rangle -f(x) , \forall x' \in \mathbb{R}^n $ \\
    i.e $ f(x) + f^\star (y) = \langle x , y \rangle $
  \end{defi}

  \begin{theorem}
    $f$ convexe propre, $x \in \ri( \dom (f) ) \Rightarrow \delta f(x) \neq \emptyset $  
  \end{theorem}

  \begin{corol}
    $f$ fonction convexe propre $ \Rightarrow $ $f$ minorée par une fonction affine.
  \end{corol}

  \begin{theorem}
    Soit $f$ fonction propre, minorée par une fonction affine. Alors $ \epi (f^{\star \star}) = \overline{\conv (\epi (f))} $
  \end{theorem}

  \begin{corol}
    Soit $f$ fonction propre. \\
    $ f = f^{\star \star} \Leftrightarrow f $ convexe et s.c.i.
  \end{corol}

  \begin{prop}
    Si $f$ est convexe s.c.i, l'image du domaine de $f$ par $ \delta f$ contient l'interieur du domaine de $f^{\star \star}$
  \end{prop}

  \section{Convexité et différentiabilité}

  \begin{defi}
    Soit $f$ fonction définie au voisinage de $x$ et soit $v \in \mathbb{R}^n$. On dit que $f$ admet une dérivée directionelle en $s$ dans la direction $v$, notée $f'(x,v)$ si :\\
    $ f'(x,v) = \lim_{t \to 0^+}\frac{f(x+t v) - f(x)}{t} $
  \end{defi}

  \begin{defi}
    $f$ Gâteaux-differentiable en $s$ si elle admet des dérivées dans toutes les directions et si il existe $y \in \mathbb{R}^n$ tel que $\forall v , f'(x,v) = \langle \nabla f(x) , v \rangle$ et on pose $f'(x) = y$.
  \end{defi}

  \begin{lemme}
    Soit $f$ une fonction convexe et $x \in \overset{\circ}{\dom (f)}$. Alors pour toute direction $v$, la dérivée directionnelle $f'(x,v)$ existe, et il existe $ y \in \diff f(x) $ telle que $f'(x,v) = \langle y,v \rangle$.
  \end{lemme}


  % You can even have references
  \rule{0.3\linewidth}{0.25pt}
  \scriptsize


  \href{http://ausset.me/cheatsheets}{http://ausset.me/cheatsheets}
\end{multicols}
\end{document}
